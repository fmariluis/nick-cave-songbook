%%%%%%%%%%%%%%%%%%%%%%%%%%%%%%%%%%%%%%%%%
% Nick Cave Songbook
% Uses Poem LaTeX Template
% Version 1.0 (2/11/2015)
%
% This template has been downloaded from:
% http://www.LaTeXTemplates.com
%
% Original author:
% Vel (vel@latextemplates.com)
%
% License:
% CC BY-NC-SA 3.0 (http://creativecommons.org/licenses/by-nc-sa/3.0/)
% Note: License applies to LaTeX code only, since I definitely don't have any rights to the lyrics!
%
% General notes:
% 1) All lines in a verse environment must end with \\, the last verse in a stanza
% must end in \\!
% 2) This template is based on the verse package, see the package documentation
% included with the template for further customisation options
% 
%%%%%%%%%%%%%%%%%%%%%%%%%%%%%%%%%%%%%%%%%

%----------------------------------------------------------------------------------------
%   DOCUMENT CONFIGURATIONS AND INFORMATION
%----------------------------------------------------------------------------------------

\documentclass[11pt, a4paper]{article} % Document font size and paper size

\usepackage{verse} % Required for typesetting poems - this package drives this template
\usepackage{titlesec}

\usepackage[T1]{fontenc} % International character encodings
\usepackage{palatino} % Use the Palatino font by default
% \usepackage{stix} % Alternative Stix font

\setlength{\parindent}{0pt} % Disable paragraph indentation

% Author styles
\newcommand{\poemauthorcenter}[1]{\nopagebreak{\centering\footnotesize\textsc{#1}\par}} % Author as a footnote at the end of the poem center aligned
\newcommand{\poemauthorright}[1]{\nopagebreak{\raggedleft\footnotesize\textsc{#1}\par}} % Author as a footnote at the end of the poem aligned right

\renewcommand{\poemtitlefont}{\normalfont\bfseries\large\centering} % Define the poem title style

\setlength{\stanzaskip}{0.75\baselineskip} % The distance between stanzas

\pagestyle{empty} % Stop page numbering through the document

\titleformat{\chapter}[display]
{\normalfont\Large\filcenter\sffamily}
{\vspace*{\fill}
 \titlerule[1pt]%
 \vspace{1pt}%
 \titlerule
 \vspace{1pc}%
 \LARGE\MakeUppercase{\chaptertitlename}~\thechapter}
{1pc}
{\titlerule\Huge}
[\vspace*{\fill}\newpage]

\begin{document}

\chapter{And No More Shall We Part - 2001}
\newpage

%----------------------------------------------------------------------------------------
% As I sat sadly by her side
%----------------------------------------------------------------------------------------

\poemtitle{As I sat sadly by her side}
\poemauthorright{Nick Cave and the Bad Seeds} % Centered author

\settowidth{\versewidth}{Because it was grassy and wanted wear;} % Insert one of the average-sized verses, used for centering the poem

\begin{verse}[\versewidth]

As I sat sadly by her side \\
At the window, through the glass \\
She stroked a kitten in her lap \\
And we watched the world as it fell past \\
Softly she spoke these words to me \\
And with brand new eyes, open wide \\
We pressed our faces to the glass \\
As I sat sadly by her side \\!

She said, \textquotedblleft Father, mother, sister, brother, \\
Uncle, aunt, nephew, niece, \\
Soldier, sailor, physician, labourer, \\
Actor, scientist, mechanic, priest \\
Earth and moon and sun and stars \\
Planets and comets with tails blazing \\
All are there forever falling \\
Falling lovely and amazing\textquotedblright \\!

Then she smiled and turned to me \\
And waited for me to reply \\
Her hair was falling down her shoulders \\
As I sat sadly by her side \\!

As I sat sadly by her side \\
The kitten she did gently pass \\
Over to me and again we pressed \\
Our different faces to the glass \\
“That may be very well”, I said \\
“But watch the one falling in the street \\
See him gesture to his neighbours \\
See him trampled beneath their feet \\
All outward motion connects to nothing \\
For each is concerned with their immediate need \\
Witness the man reaching up from the gutter \\
See the other one stumbling on who can not see” \\!

With trembling hand I turned toward her \\
And pushed the hair out of her eyes \\
The kitten jumped back to her lap \\
As I sat sadly by her side \\!

Then she drew the curtains down \\
And said, “When will you ever learn \\
That what happens there beyond the glass \\
Is simply none of your concern? \\
God has given you but one heart \\
You are not a home for the hearts of your \\
brothers \\!

And God does not care for your benevolence \\
Anymore than he cares for the lack of it in others \\
Nor does he care for you to sit \\
At windows in judgement of the world He created \\
While sorrows pile up around you \\
Ugly, useless and over-inflated” \\!

At which she turned her head away \\
Great tears leaping from her eyes \\
I could not wipe the smile from my face \\
As I sat sadly by her side \\

\end{verse}

\newpage

%----------------------------------------------------------------------------------------
%	And no more shall we part
%----------------------------------------------------------------------------------------

\poemtitle{And no more shall we part}
\poemauthorright{Nick Cave and the Bad Seeds} % Centered author

\settowidth{\versewidth}{Because it was grassy and wanted wear;} % Insert one of the average-sized verses, used for centering the poem

\begin{verse}[\versewidth]

And no more shall we part \\
It will no longer be necessary \\
And no more will I say, dear heart \\
I am alone and she has left me \\!

And no more shall we part \\
The contracts are drawn up, the ring is locked \\
upon the finger \\
And never again will my letters start \\
Sadly, or in the depths of winter \\!

And no more shall we part \\
All the hatchets have been buried now \\
And all of birds will sing to your beautiful heart \\
Upon the bough \\!

And no more shall we part \\
Your chain of command has been silenced now \\
And all of those birds would've sung to your \\
beautiful heart \\
Anyhow \\!

Lord, stay by me \\
Don't go down \\
I will never be free \\
If I'm not free now \\!

Lord, stay by me \\
Don't go down \\
I never was free \\
What are you talking about? \\!

For no more shall we part \\
And no more shall we part \\

\end{verse}
\newpage

\end{document}